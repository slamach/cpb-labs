\tableofcontents

\newpage

\section{Задание}
По выданному преподавателем варианту разработать и исследовать работу комплекса программ обмена данными в режиме прерывания программы. Основная программа должна изменять содержимое заданной ячейки памяти (Х), которое должно быть представлено как знаковое число. Область допустимых значений изменения Х должна быть ограничена заданной функцией F(X) и конструктивными особенностями регистра данных ВУ (8-ми битное знаковое представление). Программа обработки прерывания должна выводить на ВУ модифицированное значение Х в соответствии с вариантом задания, а также игнорировать все необрабатываемые прерывания.
\begin{figure}[H]
\centering
\includegraphics[scale=0.5]{task}
\label{pic:task}
\end{figure}

\section{Текст комплекса программ}
\begin{center}
\begin{tabular}{c}
\begin{lstlisting}[basicstyle=\ttfamily]
	ORG 0x0
V0:	WORD $DEF, 0x180
V1:	WORD $DEF, 0x180
V2:	WORD $INT2, 0x180
V3:	WORD $INT3, 0x180
V4:	WORD $DEF, 0x180
V5:	WORD $DEF, 0x180
V6:	WORD $DEF, 0x180
V7:	WORD $DEF, 0x180
DEF:	IRET

	ORG 0x01D
X:	WORD -10
X_MAX:	WORD 19
X_MIN:	WORD -17

START:	DI
	LD #0xA
	OUT 5
	LD #0xB
	OUT 7

CYCLE:	HLT;	BREAKPOINT 1
	DI
	LD X
	SUB #2
	CMP X_MIN
	BGE STORE
	LD X_MAX
STORE:	ST X
	HLT;	BREAKPOINT 2
	EI
	BR CYCLE
\end{lstlisting}
\end{tabular}
\end{center}

\begin{center}
\begin{tabular}{c}
\begin{lstlisting}[basicstyle=\ttfamily]
INT2:	CLA
	IN 0x4
	AND #0x0F
	AND $X
	ST $X
	IRET

INT3:	LD $X
	ASL
	ASL
	ASL
	SUB $X
	NEG
	ADD #6
	OUT 0x6
	IRET
\end{lstlisting}
\end{tabular}
\end{center}

\section{Описание комплекса программ}
\subsection{Назначение комплекса программ}
Основная программа уменьшает на 2 содержимое X (ячейки памяти с адресом 0x01D) в цикле. Если значение оказывается вне ОДЗ, в X помещается максимальное по ОДЗ число. По нажатию кнопки готовности КВУ-2 обработчик прерывания выполняет операцию побитового маскирования, оставляя 4 младших разряда содержимого регистра данных КВУ-2 и X, результат записывается в X. По нажатию кнопки готовности КВУ-3 обработчик прерывания осуществляет вывод результата вычисления функции $F(X)=-7X+6$ на КВУ-3.

\subsection{Область представления и область допустимых значений данных}
\subsubsection{Область представления данных}
\noindent Числа X, X\_MAX, X\_MIN: 8-разрядные знаковые целые числа\\
(для хранения в памяти БЭВМ используется расширение знака)\\
Содержимое регистра данных КВУ-2: набор из 8 логических значений

\subsubsection{Область допустимых значений данных}
\noindent ОДЗ X ограничена функцией $F(X)=-7X+6$ и 8-битным знаковым представлением РДВУ-3.\\

\noindent -17 (0xEF) $\leqslant$ X $\leqslant$ 19 (0x13)\\
X\_MAX $=const=$ 19 (0x0013)\\
X\_MIN $=const=$ -17 (0xFFEF)

\subsection{Расположение в памяти ЭВМ}
\noindent Основная программа: 020\ldots02F\\
Обработчик прерывания КВУ-2: 030\ldots035\\
Обработчик прерывания КВУ-3: 036\ldots03E\\
Обработчик прерывания по умолчанию: 010\\

\noindent Адрес переменной: 01D (X)\\
Адрес максимального значения переменной: 01E (X\_MAX)\\
Адрес минимального значения переменной: 01F (X\_MIN)

\subsection{Адреса первой и последней выполняемой команд основной программы}
\noindent Адрес первой команды основной программы: 020\\

\section{Методика проверки}
X = -10 (0xFFF6), X\_MAX = 19 (0x0013), X\_MIN = -17 (0xFFEF)
\begin{itemize}
\item Загрузить исходные данные и комплекс программ в пямять БЭВМ
\item Убедиться, что в точках останова по адресу 025 и 02D установлено HLT
\item Запустить основную программу в режиме работы с адреса 020 и дождаться останова
\item Программа остановится перед первой итерацией цикла уменьшения переменной X
\item Произвести пуск еще раз, чтобы выполнить уменьшение переменной, и еще раз, чтобы остановиться перед следующим циклом
\item Повторить предыдущий пункт еще 2 раза
\item Прочитать значение ячейки X (01D) и убедиться, что там находится значение -16 (0xFFF0)
\item Вернуть в счетчик команд адрес 026 и произвести очередной пуск
\item Прочитать значение ячейки X (01D) и убедиться, что там находится значение 19 (0x13)
\item Установить значение 0xFF в регистр данных КВУ-2 и нажать кнопку готовности
\item Вернуть в счетчик команд адрес 02E и произвести очередной пуск
\item Прочитать значение ячейки X (01D) и убедиться, что там находится значение 3 (0x3)
\item Вернуть в счетчик команд адрес 026 и произвести очередной пуск
\item Прочитать значение ячейки X (01D) и убедиться, что там находится значение 1 (0x1)
\item Нажать кнопку готовности КВУ-3
\item Вернуть в счетчик команд адрес 02E и произвести очередной пуск
\item Посмотреть на значение регистра данных КВУ-3 и убедиться, что там находится значение -1 (0xFF)
\item Порадоваться, что все работает
\end{itemize}

\section{Вывод}
В ходе выполнения данной лабораторной работы я познакомился с работой прерываний в БЭВМ, векторами прерывания и новыми для меня командами - DI, EI, IRET. Эти знания пригодятся мне для дальнейшей работы с БЭВМ и понимания работы современных ЭВМ.
