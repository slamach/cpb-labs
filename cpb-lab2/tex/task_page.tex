\section{Задание}
По выданному преподавателем варианту определить функцию, вычисляемую программой, область представления и область допустимых значений исходных данных и результата, выполнить трассировку программы, предложить вариант с меньшим числом команд. При выполнении работы представлять результат и все операнды арифметических операций знаковыми числами, а логических операций набором из шестнадцати логических значений.
\begin{figure}[H]
\centering
\includegraphics[scale=0.6]{task_test}
\label{pic:task}
\end{figure}

\section{Исходная программа}
\begin{center}
\begin{tabular}{|c|c|c|l|}
\hline
\textbf{Адрес ячейки} & \textbf{Содержимое ячейки} & \textbf{Мнемоника} & \textbf{Описание}\\
\hline
94 & 30A0 & --- & Исходные данные\\
95 & 0200 & --- & Результат\\
\hline
96 & A094 & LD 94 & Записать в AC значение ячейки 94\\
97 & 609F & SUB 9F & Вычесть из AC значение ячейки 9F\\
98 & E09E & ST 9E & Записать значение AC в ячейку 9E\\
99 & 0200 & CLA & Очистить AC\\
9A & 30A0 & OR A0 & Побитовое `ИЛИ' значения AC и ячейки A0\\
9B & 309E & OR 9E & Побитовое `ИЛИ' значения AC и ячейки 9E\\
9C & E095 & ST 95 & Записать значение AC в ячейку 95\\
9D & 0100 & HLT & Перейти в режим останова\\
\hline
9E & 0200 & --- & Промежуточные данные\\
9F & 30A0 & --- & Исходные данные\\
A0 & 30A0 & --- & Исходные данные\\
\hline
\end{tabular}
\end{center}

\section{Описание программы}
\subsection{Назначение и реализуемая функция}
Программа реазиует побитовую операцию `ИЛИ' для значения ячейки A0 и разности ячеек 94 и 9F и записывает ее результат в ячейку 95. Обозначим число в ячейке памяти 95 за $R$, число в 94 за $X$, в 9F за $Y$, а в A0 за $Z$. Получим следующую формулу:
$$R=(X-Y) \vee Z$$

\subsection{Область представления}
\noindent Ячейки $94, 9E, 9F, A0$ --- 16-разрядные знаковые числа\\
Ячейка $95$ --- 16-битовый результат операции `ИЛИ'

\subsection{Область допустимых значений}
\noindent Ячейки $94, 9E, 9F, A0$: $-2^{15}\ldots2^{15}-1$

\subsection{Расположение программы в памяти}
\noindent Исходные данные: 094, 09F, 0A0\\
\noindent Промежуточные данные: 09E\\
\noindent Промежуточные данные: 09E\\
\noindent Программа: 096-09D\\
\noindent Результат: 095

\subsection{Адреса первой и последней команды программы}
\noindent Адрес первой команды: 096\\
\noindent Адрес последней команды: 09D

\section{Таблица трассировки}
\begin{center}
\begin{tabular}{|c|c|c|c|c|c|c|c|c|c|c|c|}
\hline
\multicolumn{2}{|c}{\makecell{\textbf{Выполняемая}\\\textbf{команда}}}
  &\multicolumn{8}{|c|}{\textbf{Содердимое регистров после выполнения команды}}
  &\multicolumn{2}{c|}{\makecell{\textbf{Ячейка, содержимое}\\\textbf{которой изменилось}}}\\
\hline
Адрес & Код & IP & CR & AR & DR & SP & BR & AC & NZVC & Адрес & Новый код\\
\hline
096 & A094 & 097 & A094 & 094 & 30A0 & 000 & 0096 & 30A0 & 0000 & --- & ---\\
\hline
097 & 609F & 098 & 609F & 09F & 30A0 & 000 & 0097 & 0000 & 0101 & --- & ---\\
\hline
098 & E09E & 099 & E09E & 09E & 0000 & 000 & 0098 & 0000 & 0101 & 09E & 0000\\
\hline
099 & 0200 & 09A & 0200 & 099 & 0200 & 000 & 0099 & 0000 & 0101 & --- & ---\\
\hline
09A & 30A0 & 09B & 30A0 & 0A0 & 30A0 & 000 & CF5F & 30A0 & 0001 & --- & ---\\
\hline
09B & 309E & 09C & 309E & 09E & 0000 & 000 & CF5F & 30A0 & 0001 & --- & ---\\
\hline
09C & E095 & 09D & E095 & 095 & 30A0 & 000 & 009C & 30A0 & 0001 & 095 & 30A0\\
\hline
09D & 0100 & 09E & 0100 & 09D & 0100 & 000 & 009D & 30A0 & 0001 & --- & ---\\
\hline
\end{tabular}
\end{center}

\section{Вариант программы с меньшим числом команд}
\begin{center}
\begin{tabular}{|c|c|c|l|}
\hline
\textbf{Адрес ячейки} & \textbf{Содержимое ячейки} & \textbf{Мнемоника} & \textbf{Описание}\\
\hline
94 & X & --- & Исходные данные\\
95 & Y & --- & Исходные данные\\
96 & Z & --- & Исходные данные\\
97 & R & --- & Результат\\
\hline
98 & A094 & LD 94 & Записать в AC значение ячейки 94\\
99 & 6095 & SUB 95 & Вычесть из AC значение ячейки 95\\
9A & 3096 & OR 96 & Побитовое `ИЛИ' значения AC и ячейки 96\\
9B & E097 & ST 97 & Записать значение AC в ячейку 97\\
9C & 0100 & HLT & Перейти в режим останова\\
\hline
\end{tabular}
\end{center}

\section{Вывод}
В ходе данной лабораторной работы я познакомился с утройством БЭВМ и ее командами. Научился обращаться к памяти, исполнять простейшие программы и исправлять в них ошибки. Данные знания дают базовое понимаю работы современных ЭВМ.
